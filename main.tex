\documentclass{qx-files/qx-notes}

\usepackage{catppuccinpalette}



% IDIOMA ========================================================
\usepackage[spanish, mexico]{babel}


% DATOS =========================================================
\title{Plantilla Notas}
\author{Qx}
\date{\today}


\begin{document}
  \maketitle


  \makeatletter
  \texttt{\ifqx@pagecolor Verdadero\else falso\fi}

  \texttt{|\meaning\qx@themeName|}

  \expandafter\ifdefined\csname color@qx-text\endcsname
    \texttt{\expandafter\meaning\csname \string\color@qx-text\endcsname}%
  \else no defined\fi

  l\ifblank{\qx@themeName}{empty}{\qx@themeName}l

  \makeatother


  \chapter{Introducción}

  Esta plantilla estará destinada a la creación de notas para diversos usos, por ejemplo para código o para matemáticas.

  Tendrá todos los comando que normalmente usaría y estará descrito en varios paquetes que podrán ser usado en otros contextos.

  \section{Clases y paquete}

  La plantilla define multiples clases y paquetes. Esta sección describe todos los diversos documentos, los cuaes estarán guardados en la carpeta \verb|qx-files|.

  \subsection{Clase qx-notes}

  En primer lugar, tenemos la clase \verb|qx-notes|. Este puede ser accedido mediante
  \begin{verbatim}
    \documentclass{qx-files/qx-notes}
  \end{verbatim}

  Esta clase es la clase principal de la plantilla. Es donde estarán definidas todos los entornos y comandos que conciernen a el diseño de la plantilla. Por ejemplo la definición de las cajas o el listing.

  Tiene alguns opciones, por ejemplo tenemos \verb|set-theme| el cual determina el tema y colores que serán usados por el documento. Se tienen los temas \verb|default|, \verb|dark| y \verb|monokai| y los definidos por el paquete \verb|catppuccin|.

  La siguiente opción será \verb|page-size|, este define el comportamiento de la página. Tendrá las opciones \verb|letter|, \verb|a4| y \verb|a5| para tamaños de página comunes. Pero también incluye la opción \verb|phone| para un tamaño de página que sea facilmente visible en celulares.


  \section{Dependencias}

  Este paquete usará tcolorbox con la librería de minted para la generación de los listings. Minted usa la herramienta externa de Pygments, por lo que en ciertas distribuciones es necesario activar la opción \verb|--shell-escape| y tener Python instalado con la librería Pygments.

  También, para el color se usará la librería de catpuccin. Esta puede ser descargada mediante pip





\end{document}
