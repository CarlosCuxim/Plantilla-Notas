\documentclass[theme=mocha, pagecolor=true, pagesize=mobile]{qx-files/qx-notes}



% IDIOMA ========================================================
\usepackage[spanish, mexico]{babel}



% DATOS =========================================================
\title{Plantilla Notas}
\author{Qx}
\date{\today}



\usepackage{lipsum}


\begin{document}
  \maketitle



  \section{Introducción}

  Esta plantilla estará destinada a la creación de notas para diversos usos, por ejemplo para código o para matemáticas.

  Tendrá todos los comando que normalmente usaría y estará descrito en varios paquetes que podrán ser usado en otros contextos.



  \section{Clases y paquete}

  La plantilla define multiples clases y paquetes. Esta sección describe todos los diversos documentos, los cuaes estarán guardados en la carpeta \verb|qx-files|.



  \subsection{Clase qx-notes}

  En primer lugar, tenemos la clase \verb|qx-notes|. Este puede ser accedido mediante
  \begin{verbatim}
    \documentclass{qx-files/qx-notes}
  \end{verbatim}

  Esta clase es la clase principal de la plantilla. Es donde estarán definidas todos los entornos y comandos que conciernen a el diseño de la plantilla. Por ejemplo la definición de las cajas o el listing.

  Tiene alguns opciones, por ejemplo tenemos \verb|set-theme| el cual determina el tema y colores que serán usados por el documento. Se tienen los temas \verb|default|, \verb|dark| y \verb|monokai| y los definidos por el paquete \verb|catppuccin|.

  La siguiente opción será \verb|page-size|, este define el comportamiento de la página. Tendrá las opciones \verb|letterpaper|, \verb|a4paper| y \verb|a5paper| para tamaños de página comunes. Pero también incluye la opción \verb|mobile| para un tamaño de página que sea facilmente visible en celulares (probado en un poco phone x3).



  \section{Dependencias}

  Este paquete usará tcolorbox con la librería de minted para la generación de los listings. Minted usa la herramienta externa de Pygments, por lo que en ciertas distribuciones es necesario activar la opción \verb|--shell-escape| y tener Python instalado con la librería Pygments.

  También, para el color se usará la librería de catpuccin. Esta puede ser descargada mediante pip


  \section{listings examples}

  La clase \verb|qx-notes| genera varios entornos para colorcar código.

  En primer lugar tenemos el entorno \verb|{codeblock}| el cual nos permite crear una caja simple donde el codigo puesto dentro se va a renderizar con una fuente monoespaciada y con la sintaxis coloreada mediante el paquete \verb|{minted}|.

  Tiene dos argumentos opcionales, el primero encerrado en los tokens \verb|<| y \verb|>| definirá el lenguaje del código. Si ningún valor se da, entonces se usará \verb|latex|. Esto puede ser cambiado con el siguiente comando
  \begin{texcode}
    \qxsetmainlanguage{<lang>}
  \end{texcode}
  
  El segundo argumento opcional, encerrado en los clásicos corchetes, se podrá usar para colocar opciones adicionales al tcolorbox.

  \begin{texexample}
    \begin{codeblock}<c>[colframe=red]
      int main() {
        printf("hello, world");
        return 0;
      }
      // Hola
    \end{codeblock}
  \end{texexample}

  También se define los entornos \verb|texcode| y \verb|texexample|, la principal diferencia es que el primer incluye el rederizado del código, mientras que el segundo es solo el ćodigo.

  \begin{codeblock}[listing and text]
    \begin{texexample}
      El teorema de pitágoras dice que
      \[ a^2 + b^2 = c^2 \]
    \end{texexample}
  \end{codeblock}

  \begin{texexample}
    \begin{texcode}
      Este ejemplo fue hecho con el entorno
      \verb|{texcode}|
    \end{texcode}
  \end{texexample}


  \section{lipsum text}

  \lipsum

\end{document}
