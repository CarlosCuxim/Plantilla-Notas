\documentclass[theme=mocha, pagecolor, pagesize=a5paper, stretchmode=true]{qx-files/qx-notes}



% IDIOMA ========================================================
\usepackage[spanish, mexico]{babel}





% DATOS =========================================================
\title{\texpkgname{qx-files}}
\author{Qx}
\date{\today}



\usepackage{lipsum}


\begin{document}
  \maketitle




  \begin{abstract}
    El objetivo de esta plantilla es definir una serie de macros y herramientas con el objetivo de la creación de notas matemáticas o de programación de manera simple.

    Incluye varias opciones para la customización de la apariencia.También, estará modularizado, por lo que si solo se requiere una porción de los comandos, basta con cargar el paquete que lo contenga únicamente.
  \end{abstract}



  \section{Clases y paquete}

  La plantilla define multiples clases y paquetes. Esta sección describe todos los diversos documentos, los cuaes estarán guardados en la carpeta \codeline{qx-files}.



  \subsection{Clase \texpkgname{qx-notes}}

  En primer lugar, tenemos la clase \texpkgname{qx-notes}. Este puede ser accedido mediante el siguiente comando
  \begin{texcode}
    \documentclass{qx-files/qx-notes}
  \end{texcode}

  Esta clase es la principal de la plantilla. Es donde estarán definidas todos los entornos y comandos que conciernen a el diseño de esta. Por ejemplo la definición de las cajas o el listing.

  Tiene alguns opciones, por ejemplo tenemos \codeline{theme} el cual determina el tema y colores que serán usados por el documento. Se tienen los temas \codeline{default}, \codeline{dark} y \codeline{monokai} y los definidos por el paquete \texpkgname{catppuccin}.

  La siguiente opción será \codeline{pagesize}, este define el comportamiento de la página. Tendrá las opciones \codeline{letterpaper}, \codeline{a4paper} y \codeline{a5paper} para tamaños de página comunes. Pero también incluye la opción \codeline{mobile} para un tamaño de página que sea facilmente visible en celulares (probado en un poco phone x3).


  \section{listings examples}

  La clase \texpkgname{qx-notes} genera varios entornos para colorcar código.

  En primer lugar tenemos el entorno \texenvname{codeblock} el cual nos permite crear una caja simple donde el codigo puesto dentro se va a renderizar con una fuente monoespaciada y con la sintaxis coloreada mediante el paquete \texpkgname{minted}.

  Tiene dos argumentos opcionales, el primero encerrado en los tokens \codeline{<} y \codeline{>} definirá el lenguaje del código. Si ningún valor se da, entonces se usará \codeline{text} (texto plano). Esto puede ser cambiado con el siguiente comando
  \begin{texcode}
    \qxsetmainlanguage{<lang>}
  \end{texcode}
  
  El segundo argumento opcional, encerrado en los clásicos corchetes, se podrá usar para colocar opciones adicionales al tcolorbox.

  \begin{texexample}
    \begin{codeblock}<c>[colframe=red]
      int main() {
        printf("hello, world");
        return 0;
      }
      // Hola
    \end{codeblock}
  \end{texexample}

  También se define los entornos \texenvname{texcode} y \texenvname{texexample}, la principal diferencia es que el primer incluye el rederizado del código, mientras que el segundo es solo el ćodigo.

  \begin{codeblock}[listing and text]
    \begin{texexample}
      El teorema de pitágoras dice que
      \[ a^2 + b^2 = c^2 \]
    \end{texexample}
  \end{codeblock}

  \begin{texexample}
    \begin{texcode}
      Ejemplo de código en \LaTeX\ sin renderizar.
    \end{texcode}
  \end{texexample}

  Finalmente tenemos los comandos de código para su uso dentro del texto. Primero tenemos a \texcs\codeline. Al igual que \texenvname{block} toma dos argumentos opcionaes, el lenguaje entre \codeline{<} y \codeline{>} y las opciones generales entre corchetes. Sin embargo, para colocar el código, puede hacerce entre dos token, como el comando \texcs\verb\ o etre llaves.

  \begin{texexample}
    En python se puede hacer un bucle infinito con 
    \codeline<py>{while True:}.

    Pero en C hay que usar \codeline<c>|while(true){}|.
  \end{texexample}


  En el caso de código para \TeX/\LaTeX\ existen varios comandos. El primero es \texcs\texline\ el cual es un alias de \texline{\codeline<latex>}. Es posible crear más de estos alias mediante el comando \texcs\newmintinline, el cual se detalla en la documentación de \texpkgname{codeline}.

  Para simplificar algunas cosas. También es posible usar los comandos \texcs\texenvname\ y \texcs\texpkgname\ que escribe el nombre de un entorno o paquete con el formato correspondiente. Finalmene tenemos el comando \texcs\texcs\ el cual convierte un comando a su forma escrita (es simplemente para eliminar la necesidad de colocar llaves).

  \begin{texexample}
    Los comandos \texcs\mint\ así como el entorno
    \texenvname{minted} son parte del paquete \texpkgname{minted}.
  \end{texexample}

  En el caso de estar escribiendo tutoriales, usualmente es necesario definir parámetros de entrada o botones. Para ello existen los comandos \texcs\inputparameter\ y \texcs\keybutton. 

  \begin{texexample}
    Una entrada podría ser \inputparameter{entrada} mientras que un botón podría ser \keybutton{botón}.
  \end{texexample}

  \section{Dependencias}

  El entorno \texpkgname{qx-notas} usará \texpkgname{tcolorbox} con la librería de \texpkgname{minted} para la generación de los listings. \texpkgname{minted} usa la herramienta externa de Pygments, por lo que en ciertas distribuciones es necesario activar la opción \codeline{--shell-escape} y tener Python instalado con la librería Pygments.

  También, para el color se usará la librería de catpuccin. Esta puede ser descargada mediante \codeline{pip}.




  \section{lipsum text}

  \lipsum

\end{document}
